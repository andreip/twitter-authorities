\chapter{Experiments and Results}
\label{sec:results}

This chapter is dedicated to detailing more on the final step from line \ref{alg:get_top_users} in Algorithm \ref{alg:authority} (page \pageref{alg:authority}). This step of using the resulted features in order to obtain a top of authority users can be done in multiple ways:
\begin{enumerate}
	\item Using a non-weighted mean over the scaled features
    \item Using a weighted mean over the scaled features
    \item Using some clustering algorithms and decide:
    \begin{itemize}
    	\item what to do with the clusters
        \item which cluster(s) to pick as final
        \item which users to pick from the final cluster(s)
    \end{itemize}
\end{enumerate}

Taking the above steps, we propose followup sections for each, treat them and show results obtained.

\section{Non-Weighted Features Mean}

Using a \codeblock{mean(features)} approach, we got the following results from table \ref{table:res1}. What \textbf{good users} like WTA, Wimbledon etc. have in common is that they are well above the median with the features:
\begin{itemize}
	\item Mention Impact: WTA - 12.54, Wimbledon - 364.5
    \item Retweet Impact: WTA - 26, Wimbledon - 94.18
\end{itemize}

So notice that these two are incredibly popular among tennis lovers and these metrics are highly useful for promoting these two. Spammers have absolute \textbf{0} on these ones, the lowest possible value.

We detected the spams/bots from table \ref{table:res1} using an external app \textbf{BotOrNot}\footnote{\label{fn:botornot}http://truthy.indiana.edu/botornot/} developed at the Indiana University. I asked whether they have an API available and responded that it's still in development and testing. This tool\footref{fn:botornot} is indeed useful for removing spammers from the list of top users and will help in improving the list significantly when an API will be available.

\subsection{With URL feature}

The following results from table \ref{table:res1} are obtained using all features from Section \ref{sec:features}.

\begin{table}[!h]
\centering
\setlength{\tabcolsep}{12pt}
\begin{tabular}{| c | c | c |}
\hline
Query & "halep" & "ukraine gas russia" \\
\hline
Users & 2,736 & 10,967 \\
\hline
Users for which & 481 & 2,266 \\
computed features & & \\
\hline
Percent & 18\% & 20\% \\
\hline
\# Authorities & 13 & 16\\
\hline
Good Users & Wimbledon,    		  & RT\_com, Reuters \\
		   & WTA,                 & CNNMoney, KyivPost \\
           & HalepFans            & MaximEristavi, guardian \\
           &					  & STForeignDesk \\
\hline
Good (\%) & 23\% & 44\% \\
\hline
Spam/Bot Users & TennisWorld3, 					& AdeyeyeFaleye, \\
               & yoursportman,					& CollectedN\\
               & Tennisfansclub1,		   		& WatchingaBuzz \\
               &								& newzlasz \\


\hline
Spam/Bot (\%) & 23\% & 25\% \\
\hline
\end{tabular}
\caption{Query results with URL feature.}
\label{table:res1}
\end{table}

\subsection{Without URL feature}

The following results from table \ref{table:res2} are obtained using all features from Section \ref{sec:features} except the URL feature.

\begin{table}[!h]
\centering
\setlength{\tabcolsep}{12pt}
\begin{tabular}{| c | c | c |}
\hline
Query & "halep" & "ukraine gas russia" \\
\hline
Users & 2,736 & 10,967 \\
\hline
Users for which & 481 & 2,266 \\
computed features & & \\
\hline
Percent & 18\% & 20\% \\
\hline
\# Authorities & 13 & 16\\
\hline
Good Users & Wimbledon,    		  & RT\_com, Reuters \\
		   & WTA,                 & CNNMoney, KyivPost \\
           & HalepFans            & MaximEristavi, guardian \\
           &					  & STForeignDesk \\
\hline
Good (\%) & 23\% & 44\% \\
\hline
Spam/Bot Users & TennisWorld3, 					& AdeyeyeFaleye, \\
               & yoursportman,					& CollectedN\\
               & Tennisfansclub1,		   		& WatchingaBuzz \\
               &								& newzlasz \\

\hline
Spam/Bot (\%) & 23\% & 25\% \\
\hline
\end{tabular}
\caption{Query results without URL feature.}
\label{table:res2}
\end{table}

\subsection{Conclusions}

Conclusions per feature to decide on this first experiment:

\begin{description}
	\item[Mention Impact] highly useful in identifying authorities
    \item[Retweet Impact] highly useful in identifying authorities
    \item[Topical Signal] is not useful at all, because:
    \begin{itemize}
    	\item for users like "AdeyeyeFaleye", the topical signal is 0.25, compared to others which have the Topical Signal < $10^{-3}$
        \item this happens because "AdeyeyeFaleye" has few tweets posted, compared to other influential users who have a lot of tweets
    \end{itemize}
    \item[Non-Chat Signal, Signal Strength] close to one or one for all users (after scaling), but somewhat useful for avoiding users that have low Signal Strength or Non-Chat Signal
    \item[URL] As we can see, no visible differences in percentages have been obtained by using or stop using the URL feature alone. We cannot thus say anything about its usefulness.
\end{description}

\section{Weighted Features Mean}

\subsection{"halep" dataset}

The table \ref{table:weight-halep} represents various cases of applying weights and seeing how the spam/not spam percentages vary. In the first row we have the case where Retweet Impact has weight one, and all other features have weight zero.
The users found (noted here for reference) are:
\begin{description}
	\item[row 1] only Retweet Impact weight one, all others have weights zero\\
    Good Users: Wimbledon, WTA, tikitaka\_ro, SuperSportBlitz, bbctennis, BenRothenberg, HalepFanpage, SI\_Tennis, Tennis, ESPNTennis, MindTheRacket, HalepMania, adidastennis
    \item[row 2] Retweet Impact and Mention Impact both have same weight, all others have weight 0\\
    Good Users: same as row 1, but instead of adidastennis, we got livetennis
    \item[row 3]
    Good Users: 1stTennisNews, AP\_Sports, ASBandara
\end{description}


\begin{table}[!h]
\centering
\setlength{\tabcolsep}{12pt}
\begin{tabular}{| c | c | c | c | c | c |}
\hline
Weights 	& Row 	& Good (\%) & Spam (\%) \\
\hline
RI=1		& 1		& 100\%		& 0\%		\\
\hline
RI=1,MI=1	& 2		& 100\%		& 0\%		\\
\hline
SS=1,nCS=1	& 3		& 23\%		& 30\%		\\
\hline
SS=1,nCS=1,MI=1	& 4		& 46\%		& 15\%		\\
\hline
SS=1,nCS=1,RI=1	& 5		& 100\%		& 0\%		\\
\hline
SS=1,nCS=1,U=1	& 6		& 0\%		& 46\%		\\
\hline
SS=1,nCS=1,U=1	& 7		& 0\%		& 46\%		\\
\hline
\end{tabular}
\caption{Query weighted results for "halep" dataset.}
\label{table:weight-halep}
\end{table}

\subsection{"ukraine gas russia" dataset}


The table \ref{table:weight-ukraine} represents various cases of applying weights and seeing how the spam/not spam percentages vary. In the first row we have the case where Retweet Impact has weight one, and all other features have weight zero.
The users found (noted here for reference) are:
\begin{description}
	\item[row 1] only Retweet Impact weight one, all others have weights zero\\
    Good Users: MaximEristavi, RT\_com, KyivPost, BBCSteveR, Reuters, mchancecnn, EuromaidanPR, itvnews, AJELive, Steiner1776, FT, BlogsofWar, BBCBreaking, guardian, gbazov, RenieriArts
    \item[row 2] Retweet Impact and Mention Impact both have same weight, all others have weight 0\\
    Good Users: same as row 1, but instead found USATODAY and MoscowTimes instead of Steiner1776 and BlogsofWar
    \item[row 3]
    Good Users: like row 1, but found STForeignDesk, Jonnyhibberd, henrikholben, PressTV instead of others
\end{description}

\begin{table}[!h]
\centering
\setlength{\tabcolsep}{12pt}
\begin{tabular}{| c | c | c | c | c | c |}
\hline
Weights 	& Row 	& Good (\%) & Spam (\%) \\
\hline
RI=1		& 1		& 100\%		& 0\%		\\
\hline
RI=1,MI=1	& 2		& 100\%		& 0\%		\\
\hline
MI=1	& 3		& 100\%		& 0\%		\\
\hline
\end{tabular}
\caption{Query weighted results for "ukraine gas russia" dataset.}
\label{table:weight-ukraine}
\end{table}

Similar to experiments on "halep" dataset from table \ref{table:weight-halep}, the \textbf{Retweet Impact} and \textbf{Mention Impact} bring some very interesting and useful results.
And by combining other features with RI or MI, we get more noisy results, but maybe some that are not that popular but have good sources and public to influence. It remains a trade-off between exploration and exploitation as in hill climbing problem.

\subsection{URL Feature}
\label{subsec:url-feature}

We can see that the \textbf{URL} feature brings no influencers, but rather spam users. It may be because the URLs are shortened. It would help if we made a mode advanced URL match:
\begin{itemize}
	\item expand the shortened URLs at its maximum (it may be shortened a few times)
    \item compute approximation of equal expanded URLs, as we've encountered some URLs like\\ \codeblock{http://yoursportman.mobi/football_news_details.php?news_id=....} and these are very similar and easy to infer that it's a new feed you don't want to subscribe to; this user is actually detect as a spammer.
\end{itemize}


\subsection{Conclusions}

Conclusions per feature to decide on weighted experiment:

\begin{description}
	\item[Mention Impact] highly useful in identifying authorities, it gets the Good percentage upwards if used
	\item[Retweet Impact] highly useful in identifying authorities, it gets the Good percentage upwards if used
    \item[Signal Strength, non-Chat Signal] together they are not useful, they can't find many authorities (only 23\%).
    \item[URL] As we can see, URL feature is not bringing any influencers, but rather attracting spammers. See Section \ref{subsec:url-feature} on why the URL feature may have been wrongly computed for users and deserves another another try.
\end{description}

%
% * inspirat grafice din twitterRank si microblogs
% * facut subsection-uri pt query-uri diferite? cam ciudat
% * comparat resultate cu n-gram viewer de la google?
% * indicele silouhette per K ales, inclusiv rezultate
% * comparatie cu algoritmul bazat pe #followers mare ca in TwitterRank (Kendall Tau coefficient, Sparman Rank Order coefficient)

%\todo[aplicare diferite ponderi pe feature-uri, ce se obtine]
%\todo[eventual cu clusterizari ce se obtine daca sunt bune rezultatele]
%\todo[poze cu ce HTML scos de cod care face tabele? poate doar una]


% observatii: unii fac retweet la posturile lor (gen Wimbledon) cu mult dupa ce au publicat articolul Web. in schimb altii (WTA) fac rapid retweet, deci cu dus si intors
% scos feature TS pentru ca am obtinut useri boti sau neinfluenti cu el. (yeationat41 si AdeyeyeFaleye), ~/Desktop/ukraine2.html